\documentclass{beamer}
\usetheme{Frankfurt}
\usecolortheme{seahorse}

\title{What is LaTeX?}
\author{Kawsaar, Divya, Nicole, Logan}
\institute{NYIT}
\date{\today}

\begin{document}

% Slide 1
\begin{frame}
    \titlepage
\end{frame}

% Slide 2
\begin{frame}{What is LaTeX?}
\begin{itemize}
    \item LaTeX is a document preparation system used to create professional documents.
    \item It is especially popular in mathematics, computer science, physics, and engineering.
    \item Unlike Microsoft Word, LaTeX focuses on structure rather than visual formatting.
    \item Users write code that is compiled into a high-quality PDF.
\end{itemize}
\end{frame}

% Slide 3
\begin{frame}{The History of LaTeX}
\begin{itemize}
    \item LaTeX is based on a typesetting system called TeX.
    \item TeX was created in 1978 by computer scientist Donald Knuth.
    \item Knuth created TeX because he was unhappy with how mathematical books were being printed.
    \item In the 1980s, Leslie Lamport developed LaTeX to make TeX easier to use.
\end{itemize}
\end{frame}

% Slide 4
\begin{frame}{How LaTeX Was Developed}
\begin{itemize}
    \item TeX was powerful but difficult to use.
    \item Leslie Lamport created LaTeX as a set of macros built on top of TeX.
    \item These macros simplified document formatting.
    \item LaTeX allowed users to focus on content instead of layout.
    \item Today, LaTeX is maintained by a global open-source community.
\end{itemize}
\end{frame}

% Slide 5
\begin{frame}{How LaTeX Works}
\begin{itemize}
    \item You write plain text mixed with LaTeX commands.
    \item Commands begin with a backslash (e.g., \textbackslash section).
    \item The file is saved as a .tex file.
    \item A compiler processes the file and generates a PDF.
    \item The compiler interprets structure, formatting, and layout automatically.
\end{itemize}
\end{frame}

% Slide 6
\begin{frame}[fragile]{Example of Basic LaTeX Code}
\begin{verbatim}
\documentclass{article}

\begin{document}
\section{Introduction}
Hello World!
\end{document}
\end{verbatim}

This simple code generates a formatted document with a section title.
\end{frame}

% Slide 7
\begin{frame}{Why LaTeX is Powerful}
\begin{itemize}
    \item Produces professional, publication-quality documents.
    \item Handles complex mathematical equations easily:
    \[
        \int_0^\infty e^{-x} dx = 1
    \]
    \item Automatically numbers sections, equations, and figures.
    \item Manages references and bibliographies efficiently.
    \item Ideal for long research papers and theses.
\end{itemize}
\end{frame}

% Slide 8
\begin{frame}{What is Beamer?}
\begin{itemize}
    \item Beamer is a LaTeX document class used to create presentations.
    \item Instead of writing slides manually, you code them.
    \item Each slide is created using the frame environment.
    \item Beamer allows themes, transitions, and structured formatting.
    \item It is widely used for academic conference presentations.
\end{itemize}
\end{frame}

% Slide 9
\begin{frame}[fragile]{Basic Beamer Slide Code}
\begin{verbatim}
\begin{frame}{Slide Title}
\begin{itemize}
    \item First point
    \item Second point
\end{itemize}
\end{frame}
\end{verbatim}

Each frame becomes one slide in the final PDF.
\end{frame}

% Slide 10
\begin{frame}{Advantages and Conclusion}
\begin{itemize}
    \item LaTeX separates content from design.
    \item It ensures consistent formatting throughout a document.
    \item It is the standard tool for many academic journals.
    \item Beamer makes professional presentations possible using code.
    \item Learning LaTeX is valuable for STEM students and researchers.
\end{itemize}

\vspace{0.5cm}
\centering
Thank you!
\end{frame}

\end{document}
